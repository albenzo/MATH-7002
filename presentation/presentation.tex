% $Header: /Users/joseph/Documents/LaTeX/beamer/solutions/generic-talks/generic-ornate-15min-45min.en.tex,v 90e850259b8b 2007/01/28 20:48:30 tantau $

\documentclass{beamer}

% This file is a solution template for:

% - Giving a talk on some subject.
% - The talk is between 15min and 45min long.
% - Style is ornate.



% Copyright 2004 by Till Tantau <tantau@users.sourceforge.net>.
%
% In principle, this file can be redistributed and/or modified under
% the terms of the GNU Public License, version 2.
%
% However, this file is supposed to be a template to be modified
% for your own needs. For this reason, if you use this file as a
% template and not specifically distribute it as part of a another
% package/program, I grant the extra permission to freely copy and
% modify this file as you see fit and even to delete this copyright
% notice. 


\mode<presentation>
{
  \usetheme{Warsaw}
  % or ...

  \setbeamercovered{transparent}
  % or whatever (possibly just delete it)
}


\usepackage[english]{babel}
% or whatever

\usepackage[latin1]{inputenc}
% or whatever

\usepackage{times}
\usepackage[T1]{fontenc}
% Or whatever. Note that the encoding and the font should match. If T1
% does not look nice, try deleting the line with the fontenc.


\title{Tricolorings of Knots}

\author{Lucas Meyers}
% - Use the \inst{?} command only if the authors have different
%   affiliation.

\institute[Louisiana State University] % (optional, but mostly needed)
{Department of Mathematics\\
Louisiana State University}

\date[Short Occasion] % (optional)
{\today}

\begin{document}

\begin{frame}
  \titlepage
\end{frame}

\section{Tricoloring Knots}

\begin{frame}{Make Titles Informative. Use Uppercase Letters.}{Subtitles are optional.}
  % - A title should summarize the slide in an understandable fashion
  %   for anyone how does not follow everything on the slide itself.

  \begin{itemize}
  \item
    Use \texttt{itemize} a lot.
  \item
    Use very short sentences or short phrases.
  \end{itemize}
\end{frame}

\begin{frame}{Make Titles Informative.}

  You can create overlays\dots
  \begin{itemize}
  \item using the \texttt{pause} command:
    \begin{itemize}
    \item
      First item.
      \pause
    \item    
      Second item.
    \end{itemize}
  \item
    using overlay specifications:
    \begin{itemize}
    \item<3->
      First item.
    \item<4->
      Second item.
    \end{itemize}
  \item
    using the general \texttt{uncover} command:
    \begin{itemize}
      \uncover<5->{\item
        First item.}
      \uncover<6->{\item
        Second item.}
    \end{itemize}
  \end{itemize}
\end{frame}

\begin{frame}
  \frametitle{Knot Definition}
  
\end{frame}

\begin{frame}
  \frametitle{Unknot}
  
\end{frame}

\begin{frame}
  \frametitle{Trefoil}
  
\end{frame}

\begin{frame}
  \frametitle{Figure-Eight}
  
\end{frame}

\begin{frame}
  \frametitle{Equivalence}
  
\end{frame}

\begin{frame}
  \frametitle{Diagrams}
  
\end{frame}

\begin{frame}
  \frametitle{Reidmeister Moves}
  
\end{frame}

\begin{frame}
  \frametitle{Type I}
  
\end{frame}

\begin{frame}
  \frametitle{Type II}
  
\end{frame}

\begin{frame}
  \frametitle{Type III}
  
\end{frame}

\begin{frame}
  \frametitle{Tricoloring}
  
\end{frame}

\begin{frame}
  \frametitle{Type I}
  
\end{frame}

\begin{frame}
  \frametitle{Type II}
  
\end{frame}

\begin{frame}
  \frametitle{Type II}
  
\end{frame}

\begin{frame}
  \frametitle{Type III}
  
\end{frame}

\begin{frame}
  \frametitle{Type III}
  
\end{frame}

\begin{frame}
  \frametitle{Uncolorability of the Unknot}
  
\end{frame}

\begin{frame}
  \frametitle{Colorability of the Trefoil}
  
\end{frame}

\begin{frame}
  \frametitle{Uncolorability of the Figure-Eight Knot}
  
\end{frame}

\section{The Knot Group}

\begin{frame}
  \frametitle{Knot Group}
  
\end{frame}

\begin{frame}
  \frametitle{Knot Group of the Unknot}
  
\end{frame}

\begin{frame}
  \frametitle{Wirtinger Presentation}
  
\end{frame}

\begin{frame}
  \frametitle{Knot Group of the Trefoil}
  
\end{frame}

\begin{frame}
  \frametitle{Fox 3-Colorings}
  
\end{frame}

\begin{frame}
  \frametitle{Fox 3-Colorings of the Unknot and Trefoil}
  
\end{frame}

\begin{frame}
  \frametitle{Equivalence of Tricoloring and Fox 3-Coloring}
  
\end{frame}

\section{Consequences}

\begin{frame}
  \frametitle{Consequences}
  
\end{frame}

\begin{frame}
  \frametitle{Extending Tricolorability}
  
\end{frame}

\begin{frame}
  \frametitle{5-Coloring of the Figure-Eight Knot}
  
\end{frame}

\end{document}


