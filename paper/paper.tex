%% Revised 11/12/05.
%% Notice that anything written on a line that begins 
%% with  a % sign is ignored by LaTeX.
\documentclass[12pt]{amsart}

\usepackage{latexsym}
\usepackage{amsmath}
\usepackage{amsfonts}
\usepackage{amssymb}

\newtheorem{theorem}{Theorem}[section]
\newtheorem{lemma}[theorem]{Lemma}
\newtheorem{corollary}[theorem]{Corollary}
\newtheorem{proposition}[theorem]{Proposition}
\newtheorem{noname}[theorem]{}
\newtheorem{sublemma}{}[theorem]
\newtheorem{conjecture}[theorem]{Conjecture}

\theoremstyle{definition}
\newtheorem{definition}[theorem]{Definition}
\newtheorem{example}[theorem]{Example}

\theoremstyle{remark}
\newtheorem{remark}[theorem]{Remark}

\numberwithin{equation}{section}

\newcommand{\bb}[1]{\mathbb{#1}}

\begin{document}

\title{Colorings of Knots}


\author{Lucas Meyers}
\address{Mathematics Department\\
Louisiana State University\\
Baton Rouge, Louisiana}
\email{lmeye22@lsu.edu}

\date{\today}

\begin{abstract}
  Empty
\end{abstract}



\maketitle

\section{Introduction}
\label{introduction}

\section{Diagrams and Coloring}
\label{sec:diagrams-coloring}

\section{The Knot Group and $D_6$}
\label{sec:knot-group-d_6}



\section{$n$-colorings}
\label{sec:n-colorings}

Prior to this point we have talked about tricolourings
of knots. However there is a natural way to extend
our notion of colouring so that we may work with $n$
colors as opposed to limiting ourself to 3.

When it came to tricolouring knots we assigned arcs
either blue, red, or green. However we could just as
easily assigned numbers 0,1, or 2 to arcs and had
our condition instead be that if we have a crossing
as in figure

\textbf{Need a crossing}

That the equation $2x-y-z\equiv 0 \mod p$. We restrict
ourselves to odd primes for technical reasons.
Then a knot $K$ is $p$-colourable if we can assign
a value between $0$ and $p-1$ to each arc in such a
way that each crossing satisfies the above equation and
at least two different colors are used.

The proof that this is an invariant is precisely the same
as it is for showing that tricolourability is an invariant.

The extension of the Knot group definition of colouring is
even faster to extend. Let $G:=\pi_1(\bb{R}^3\setminus K,*)$. Then we
define an $n$-colouring of $K$ as a representation
$\rho$ of $G$ into the dihedral group $D_{2n}$. As before this
is clearly a knot invariant as there is no mention of a diagram.
Moreover the proof that the representations of $G$ in $D_{2n}$
correspond to colorings of diagrams follows similarly.

\section*{Acknowledgements}
\begin{thebibliography}{99}

\bibitem{jpsk} Kung, J.P.S., Critical problems, in {\em Matroid  
Theory} (eds. J. E. Bonin, J.G. Oxley, and B. Servatius), {\em  
Contemporary Mathematics} {\bf 197}, Amer. Math. Soc., Providence,  
1996, pp. 1--127.

\end{thebibliography}

\end{document}


