%% Revised 11/12/05.
%% Notice that anything written on a line that begins 
%% with  a % sign is ignored by LaTeX.
\documentclass[12pt]{amsart}

\usepackage{latexsym}
\usepackage{amsmath}
\usepackage{amsfonts}
\usepackage{amssymb}

\newtheorem{theorem}{Theorem}[section]
\newtheorem{lemma}[theorem]{Lemma}
\newtheorem{corollary}[theorem]{Corollary}
\newtheorem{proposition}[theorem]{Proposition}
\newtheorem{noname}[theorem]{}
\newtheorem{sublemma}{}[theorem]
\newtheorem{conjecture}[theorem]{Conjecture}

\theoremstyle{definition}
\newtheorem{definition}[theorem]{Definition}
\newtheorem{example}[theorem]{Example}

\theoremstyle{remark}
\newtheorem{remark}[theorem]{Remark}

\numberwithin{equation}{section}

\newcommand{\bb}[1]{\mathbb{#1}}

\begin{document}

\title{TriColorings of Knots}


\author{Lucas Meyers}
\address{Mathematics Department\\
Louisiana State University\\
Baton Rouge, Louisiana}
\email{lmeye22@lsu.edu}

\date{\today}

\begin{abstract}
  Empty
\end{abstract}



\maketitle

\section{Introduction}
\label{introduction}

\textbf{Citation Usage EX}~\cite[Problem 2.4.6]{jpsk}

\section{Diagrams and Coloring}
\label{sec:diagrams-coloring}

\section{The Knot Group}
\label{sec:knot-group-d_6}

The definition of 3-coloring is a perfectly valid definition for a knot
invariant. However it is defined in terms of diagrams and as such
we had to do some extra work with the Reidmeister moves to show that it
was indeed an invariant. Ideally there would be an equivalent definition
of coloring that does not involve diagrams. Then the fact that it is
an invariant would be immediate and it may be possible to better tease
out the topological property that is being captured by 3-coloring.

It turns out that there is a way to define 3-coloring in this manner
and it has to do with a construction called the Knot group. Ones first
instinct to get a group out of a knot might be to take the fundamental
group. However since all knots are copies of $S^1$ the fundamental group
for any knot is $\bb{Z}$. As such we have to take advantage of its
ambient space.

\begin{definition}[Knot Group]
  Let $K$ be a knot. Then the Knot group of $K$ is the fundamental
  group $\pi_1(S^3\setminus K,*)$.
\end{definition}

This definition makes use of the embedding of the knot into $S^3$ and
as such will not be identical for all knots. Naturally we need a way
to calculate the knot group. We can do this using what is called
the Wirtinger presentation~\cite{hatcher}.

\begin{theorem}[Wirtinger]
  Let $K$ be a knot and $D$ a diagram for $K$. Choose an
  orientation for $D$. Then let $x_1,\ldots,x_n$
  denote the arcs of $D$ and for each crossing give a relation
  $r_1\ldots, r_m$ such that $r_i \equiv (x_ix_jx_i^{-1}=x_k)$ where $x_i$ is
  the top arc of the crossing and $x_j$ is the arc underneath approaching
  $x_k$. Then the knot group of $K$ has the presentation
  \[
    \pi_1(S^3\setminus K,*)\cong \langle x_1,\ldots, x_n| \rangle
  \]
\end{theorem}

Now we give a couple of examples.

\begin{example}
  The Knot group of the unknot is
  \[
    \langle x_1 |\rangle \cong \bb{Z}
  \]
  as there are no crossings and only a single arc.
\end{example}

Followed by the trefoil knot.

\begin{example}
  The trefoil knot has three crossings and three arcs. Label it as
  shown in figure~\ref{fig:AnnTref}. Then the Wirtinger presentation
  of the knot group of the trefoil is
  \[
    \langle x_1,x_2,x_3|x_1x_2x_1^{-1}=x_3,x_3x_1x_3^{-1}=x_2,x_2x_3x_2^{-1}=x_1\rangle
  \]
  However the presentation of this group can be simplified to the
  form
  \[
    \langle x_1,x_1| x_1x_2x_1=x_2x_1x_2\rangle
  \]
\end{example}

\begin{figure}
  \caption{Annotated Trefoil}
  \label{fig:AnnTref}
\end{figure}

The knot group gives us another invariant of knots. Unfortunately
presentations of groups can be rather difficult to work with.
Thus in order to make the problem more tractable we instead look
at homomorphisms from the knot group to finite groups as these
will be determined by where we send generators. This is
how we will redefine coloring in terms of the knot group~\cite{quickfox}.

\begin{definition}[Fox 3-coloring]
  Let $K$ be a knot and $G$ the corresponding knot group. Then a
  Fox 3-coloring of $K$ is a homomorphism $\rho$ from
  $G$ to the symmetries of a triangle $D_6$. We say that
  $\rho$ is a non-trivial coloring if $\rho$ is surjective.

  Thus a knot $K$ is Fox 3-colorable if there exists a nontrivial
  Fox 3-coloring.
\end{definition}

Now a couple of examples using our prior work.

\begin{example}
  The knot group of the unknot is $\bb{Z}$. Since
  $\bb{Z}$ has only a single generator it is not possible to
  create a surjective homomorphism onto $D_6$ as $D_6$ has
  two generators.
\end{example}

\begin{example}
  As we calculated above the knot group for the trefoil is
  \[
    \langle x_1,x_2| x_1x_2x_1=x_2x_1x_2\rangle
  \]
  Similarly we can write a presentation of $D_6$ as
  \[
    \langle r,s| r^3=1, s^2=1,srsr=1\rangle
  \]
  This gives us two non-trivial Fox 3-colorings of the trefoil.
  The first sends $x_1$ to $r$ and $x_2$ to $s$ and the
  other swaps which generator is sent to which generator.
\end{example}

As was implied by the prior examples there is a relation between
Fox 3-coloring and the 3-colorability of knots. In fact it turns
out that they are exactly the same.

\begin{theorem}
  A knot $K$ has a 3-coloring if and only if there is a
  Fox 3-coloring of $K$.
\end{theorem}

A proof of this theorem can be found within~\cite{medwid}. 
The proof that is done relates two notions of coloring using
the Wirtinger presentation to marry the conditions necessary
for the existence of a 3-coloring and the existence of a Fox
3-coloring.

\section{n-coloring}
\label{sec:n-coloring}

So what are the benefits of looking at 3-coloring through
the lens of Fox 3-coloring? As we mentioned before it is
immediately clear that this is a knot invariant where as
when we defined 3-coloring with diagrams. Another benefit
is that it is more readily apparent how we can further
extend Fox 3-coloring than extending 3-coloring. From
the definition of 3-coloring there is some ambiguity that
would need to be sorted out about precisely what rules need
to change. Questions such as ``How many colors are necessary?''
or ``What would make a crossing valid configuration?''.

This is avoided with Fox 3-coloring. The only choice we made
was that we are looking at homomorphisms to $D_6$. We could
very easily have chosen any finite group and gotten the
same nice structure that we had above. The reason that
$D_6$ is the group used for Fox 3-coloring is that it is
the group of symmetries of the triangle. We can
define the Fox $n$-coloring similarly.

\begin{definition}[Fox $n$-coloring]
  Let $G$ be the knot group of a knot $K$. Then
  an $n$-coloring of $K$ is a homomorphism $\rho: G\rightarrow D_{2n}$
  where $\rho$ is called a non-trivial $n$-coloring if $\rho$ is
  surjective.
\end{definition}

We can also extend the notion of coloring that we started
with. However it does take more work. Instead of choosing
the colors that we did when originally defining 3-coloring
we could just as easily have used numbers instead. Then
instead of requiring that each crossing either use all three
colors or all the same it would be equivalent to each
crossing satisfying the equation
\[
  2x-y-z\equiv 0 \mod 3
\]
where the arcs of the crossing are labeled as in figure~\ref{fig:numcolor}

\begin{figure}
  
  \caption{Colored Crossing}
  \label{fig:numcolor}
\end{figure}

In this form it is more apparent how we could extend 3-coloring
to arbitrary $n$

\section*{Acknowledgements}
\begin{thebibliography}{99}

\bibitem{jpsk} Kung, J.P.S., Critical problems, in {\em Matroid  
Theory} (eds. J. E. Bonin, J.G. Oxley, and B. Servatius), {\em  
Contemporary Mathematics} {\bf 197}, Amer. Math. Soc., Providence,  
1996, pp. 1--127.

\bibitem{hatcher}

\bibitem{quickfox}

\bibitem{medwid}

\end{thebibliography}

\end{document}
