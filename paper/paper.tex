%% Revised 11/12/05.
%% Notice that anything written on a line that begins 
%% with  a % sign is ignored by LaTeX.
\documentclass[12pt]{amsart}

\usepackage{latexsym}
\usepackage{amsmath}
\usepackage{amsfonts}
\usepackage{amssymb}

\newtheorem{theorem}{Theorem}[section]
\newtheorem{lemma}[theorem]{Lemma}
\newtheorem{corollary}[theorem]{Corollary}
\newtheorem{proposition}[theorem]{Proposition}
\newtheorem{noname}[theorem]{}
\newtheorem{sublemma}{}[theorem]
\newtheorem{conjecture}[theorem]{Conjecture}

\theoremstyle{definition}
\newtheorem{definition}[theorem]{Definition}
\newtheorem{example}[theorem]{Example}

\theoremstyle{remark}
\newtheorem{remark}[theorem]{Remark}

\numberwithin{equation}{section}

\begin{document}

\title{Colorings of Knots}


\author{Lucas Meyers}
\address{Mathematics Department\\
Louisiana State University\\
Baton Rouge, Louisiana}
\email{lmeye22@lsu.edu}

\date{\today}

\begin{abstract}
  Empty
\end{abstract}



\maketitle

\section{Introduction}
\label{introduction}

\textbf{Citation Usage EX}~\cite[Problem 2.4.6]{jpsk}

\section{Diagrams and Coloring}
\label{sec:diagrams-coloring}

\section{The Knot Group and $D_6$}
\label{sec:knot-group-d_6}

\section{$n$-colorings}
\label{sec:n-colorings}


\section*{Acknowledgements}
\begin{thebibliography}{99}

\bibitem{jpsk} Kung, J.P.S., Critical problems, in {\em Matroid  
Theory} (eds. J. E. Bonin, J.G. Oxley, and B. Servatius), {\em  
Contemporary Mathematics} {\bf 197}, Amer. Math. Soc., Providence,  
1996, pp. 1--127.

\end{thebibliography}

\end{document}


